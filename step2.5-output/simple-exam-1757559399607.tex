\documentclass{article}
\usepackage{amsmath}
\usepackage{amsthm}

\begin{document}

\title{CS-UY 2413: Design \& Analysis of Algorithms}
\author{Prof. Lisa Hellerstein}
\date{Fall 2024 \\ New York University}

\maketitle

\section*{Homework 2}
Due 11:59pm Monday, Sep 30, New York time. \\
By handing in the homework you are agreeing to the Homework Rules; see EdStem.

Our Master Theorem: The version of the Master Theorem that we covered in class is on the last page of this homework. We won’t be covering the version of the Master Theorem in the textbook and you’re not responsible for knowing it. (But you may find it interesting!)

Reminder: For $r \neq 1$, $r^0 + r^1 + \dots + r^k = \frac{r^{k+1}-1}{r-1}$.

\begin{enumerate}
    \item For each example, indicate whether $f = o(g)$ (little-oh), $f = \omega(g)$ (little-omega), or $f = \Theta(g)$ (big-Theta). No justification is necessary.
    \begin{enumerate}
        \item $f(n) = 7(\log_2 n)^2$, $g(n) = n^{\frac{1}{2}}$
        \item $f(n) = n^2$, $g(n) = n^4$
        \item $f(n) = 2n^3$, $g(n) = 2^n$
        \item $f(n) = 3 \sum_{i=0}^n 2^i$, $g(n) = n^3$
        \item $f(n) = \log_2 n$, $g(n) = \log_{10} n + \log_e n$
    \end{enumerate}
    \item Give a formal proof of the following statement: If $f(n) \ge 0$ for all $n \in \mathbb{N}$, $g(n) \ge 0$ for all $n \in \mathbb{N}$, $f(n) = O(g(n))$, and $g(n)$ is strictly increasing and unbounded (meaning $\lim_{n \to \infty} g(n) = \infty$) then $\sqrt{f(n)} = O(\sqrt{g(n)})$.
    Use the formal definition of big-Oh in your answer. In your proof, you can use the fact that the value of $\sqrt{n}$ increases as $n$ increases.

    \item For each of the following recurrences, determine whether Our Master Theorem (on the last page of this HW) can be applied to the recurrence. If it can, use it to give the solution to the recurrence in $\Theta$ notation; no need to give any details. If not, write “Our Master Theorem does not apply.”
    \begin{enumerate}
        \item $T(n) = 2T(n/2) + n^2$
        \item $T(n) = 9T(n/3) + n$
        \item $T(n) = nT(n/2) + 1$
    \end{enumerate}
    \item Our Master Theorem can be applied to a recurrence of the form $T(n) = aT(n/b) + n^d$, where $a, b, d$ are constants with $a > 0$, $b > 1$, $d > 0$. Consider instead a recurrence of the form $T_{\text{new}}(n) = aT_{\text{new}}(n/b) + n \log_d n$ where $a > 0$, $b > 1$, $d > 1$ (and $T(1) = 1$).
    For each of the following, state whether the given property of $T_{\text{new}}$ is true. If so, explain why it is true. If not, explain why it is not true. (Even if you know the version of the Master Theorem in the textbook, don’t use it in your explanation.)
    \begin{enumerate}
        \item $T_{\text{new}}(n) = O(n^2)$ if $\log_b a < 2$
        \item $T_{\text{new}}(n) = \Omega(n^{\log_b a} \log n)$
    \end{enumerate}
    \item Consider the recurrence $T(n) = 2T(n/2) + n \log n$ for $n > 1$, and $T(1) = 1$.
    \begin{enumerate}
        \item Compute the value of $T(4)$, using the recurrence. Show your work.
        \item Use a recursion tree to attempt to solve the recurrence and get a closed-form expression for $T(n)$, when $n$ is a power of 2. (Check that your expression is correct by plugging in $n = 4$ and comparing with your answer to (a).)  Explain any difficulties encountered.
        \item Suppose that the base case is $T(2) = 5$, instead of $T(1) = 1$.  How does this change the solution?
    \end{enumerate}
    \item Consider a variation of mergesort that works as follows: If the array has size 1, return. Otherwise, divide the array into fourths, rather than in half. Recursively sort each fourth using this variation of mergesort. Then merge the first two fourths. Then merge the result with the last two fourths.
    \begin{enumerate}
        \item Write a recurrence for the running time of this variation of mergesort. It should be similar to the recurrence for ordinary mergesort. Assume $n$ is a power of 4.
        \item Apply Our Master Theorem to the recurrence to get the running time of the algorithm, in theta notation. Show your work.
    \end{enumerate}
    \item  Consider a recursive algorithm that operates on an array of size $n$, where $n$ is a power of 2. The algorithm recursively calls itself on two subarrays of size $n/2$, and then performs $n^2$ operations to combine the results. Write a recurrence relation for this algorithm's runtime.  Solve the recurrence using our Master Theorem.

    \item  Design a recursive algorithm that finds the maximum element in an array of size $n$. The algorithm should divide the array into three subarrays of size $n/3$, recursively find the maximum in each subarray, and then compare the three maximums to find the overall maximum.  Write a recurrence relation for the running time of this algorithm. Solve this recurrence using our Master Theorem.

    \item  Suppose you have a recursive algorithm that divides a problem of size n into 5 subproblems, each of size n/2.  The algorithm requires O(n) work to divide the problem and combine the results.  Write a recurrence relation for the running time and solve it using our Master Theorem.


    \item Consider the following recursive algorithm. Assume $n$ is a power of 2.
    \begin{itemize}
        \item If the array has only one element, return.
        \item Recursively sort the first half of the elements in the array.
        \item Recursively sort the second half of the elements in the array.
        \item Reverse the order of the elements in the second half.
    \end{itemize}
    \begin{enumerate}
        \item  Is this algorithm correct? Explain your answer.
        \item Write a recurrence expressing the running time of the algorithm.
        \item Apply Our Master Theorem to your recurrence. What is the running time of the algorithm, in theta notation?
    \end{enumerate}
\end{enumerate}

\section*{Theorem 0.1 (Our Master Theorem)}
Let $a, b, d, n_0$ be constants such that $a > 0$, $b > 1$, $d \ge 0$ and $n_0 > 0$. \\
Let $T(n) = aT(n/b) + \Theta(n^d)$ for when $n \ge n_0$, and $T(n) = \Theta(1)$ when $0 \le n < n_0$. Then,
\[
T(n) = \begin{cases}
    \Theta(n^d \log n) & \text{if } d = \log_b a \\
    \Theta(n^{\log_b a}) & \text{if } d < \log_b a \\
    \Theta(n^d) & \text{if } d > \log_b a
\end{cases}
\]
We assume here that $T(n)$ is a function defined on the natural numbers. We use $aT(n/b)$ to mean $a'T(\lfloor n/b \rfloor) + a''T(\lceil n/b \rceil)$ where $a', a'' > 0$ such that $a' + a'' = a$.

\end{document}