\documentclass{article}
\usepackage[utf8]{inputenc}
\usepackage{amsmath}

\begin{document}

\title{Introduction to Cell and Molecular Biology \\ BMS-UY 1003 \\ Summer 2023}
\date{}
\maketitle

Good Luck 

Feel free to use any resources (PPT slides, PubMed, google) for this quiz.
This is due on (or before) August 16 

\section*{Questions}

\begin{enumerate}
    \item The image below (which comes from a Biochemistry textbook) shows a simplified depiction of enzyme kinetics.  What crucial detail is missing from this representation that would make it more accurate?  (Hint: Consider the enzyme's structure and its interaction with the substrate.)

    \item Glucose moves from high to low concentration, across the plasma membrane, through a protein channel that is facilitated by a specific carrier protein. Which statement is true about this process?
    \begin{itemize}
        \item[A.] It requires ATP
        \item[B.] It is a form of passive transport
        \item[C.] Movement is against the concentration gradient of glucose
        \item[D.] The carrier protein is likely specific to glucose and a few closely related molecules.
    \end{itemize}

    \item Which statement is true about competitive inhibition?
    \begin{itemize}
        \item[A.] Binding of the inhibitor occurs away from the active site of the enzyme
        \item[B.] Inhibition is always irreversible
        \item[C.] Increasing the amount of substrate can reverse the inhibition
        \item[D.]  It is not relevant to the function of many pharmaceuticals.
    \end{itemize}

    \item The primary function of chaperone proteins within the proteasome is to:
    \begin{itemize}
        \item[A.] Catalyze peptide bond formation
        \item[B.] Unfold misfolded proteins to allow for degradation
        \item[C.]  Transport proteins to the Golgi apparatus
        \item[D.]  Synthesize new proteins from amino acids
    \end{itemize}

    \item During oxidative phosphorylation, the energy released from the electron transport chain is used to:
    \begin{itemize}
        \item[A.]  Pump H$^+$ ions into the mitochondrial matrix
        \item[B.]  Pump Na$^+$ ions across the plasma membrane
        \item[C.] Pump H$^+$ ions into the mitochondrial intermembrane space
        \item[D.] Directly phosphorylate ADP to ATP
    \end{itemize}

    \item How many ATP molecules (net) are produced during glycolysis per molecule of glucose?
    \begin{itemize}
        \item[A.] 36
        \item[B.] 38
        \item[C.] 2
        \item[D.] 0
    \end{itemize}

    \item  The process of beta-oxidation occurs in the:
    \begin{itemize}
        \item[A.] Cytosol
        \item[B.] Mitochondrial matrix
        \item[C.] Inner membrane of the mitochondria
        \item[D.] Nucleus
    \end{itemize}

    \item  Rotenone, a potent insecticide, inhibits an enzymatic complex located in the:
    \begin{itemize}
        \item[A.]  Cytosol
        \item[B.] Inner mitochondrial membrane
        \item[C.] Outer mitochondrial membrane
        \item[D.] Nucleus
    \end{itemize}

    \item The enzyme that catalyzes the conversion of glucose-6-phosphate to fructose-6-phosphate is:
    \begin{itemize}
        \item[A.] Phosphofructokinase
        \item[B.] Phosphohexoisomerase
        \item[C.] Hexokinase
        \item[D.] Pyruvate kinase
    \end{itemize}

    \item Which statement about the electron transport chain is false?
    \begin{itemize}
        \item[A.] It is located in the inner mitochondrial membrane
        \item[B.] Electrons are passed from one complex to another in a linear fashion
        \item[C.]  It generates a proton gradient across the inner mitochondrial membrane.
        \item[D.]  Oxygen is the final electron acceptor in aerobic respiration.
    \end{itemize}


\end{enumerate}

\end{document}