\documentclass{article}
\usepackage{amsmath}
\usepackage{graphicx}
\usepackage{listings}
\usepackage{geometry}
\geometry{a4paper, margin=1in}

\title{New York University \\ Tandon School of Engineering \\ Department of Computer Science and Engineering \\ Introduction to Operating Systems \\ Fall 2024 \\ Assignment 4 (10 points)}
\author{Your Name}
\date{\today}

\begin{document}

\maketitle

\section*{Problem 1 (3 points)}

\textbf{(a) (1 point)}  Explain the difference between a process and a thread.  Provide examples of when you might choose to use threads over processes, and vice versa.

\textbf{(b) (2 points)} Describe the concept of context switching in an operating system.  What are the key components of the process state that need to be saved and restored during a context switch?  How does context switching impact system performance?


\section*{Problem 2 (4 points)}

\textbf{(a) (2 points)}  A main() routine calls fork() twice as follows:

\begin{verbatim}
pid_t x=-11, y=-22;
x = fork();
if (x == 0) {
  y = fork();
}
\end{verbatim}

Assuming all fork() calls succeed, draw the process tree.  Clearly label each process with the values of x and y. (A visual representation would be included here in a PDF submission, showing a tree with branches and nodes labeled with the values of x and y for each process. This cannot be accurately represented in LaTeX without a graphics package and specific image inclusion).

\textbf{(b) (2 points)} Write a C program (\texttt{lab4\_problem2.c}) that creates the process tree described in part (a). The program should print the process ID (PID) and the values of x and y for each process.  Include error handling for failed fork() calls.


\section*{Problem 3 (3 points)}

Write a C program (\texttt{lab4\_problem3.c}) that simulates a simple producer-consumer problem using two threads. The producer thread generates random numbers between 1 and 100 and adds them to a shared buffer (represented as a circular buffer with a fixed size, e.g., 5). The consumer thread removes numbers from the buffer and prints them to the console.  Use appropriate synchronization mechanisms (e.g., mutexes, condition variables) to prevent race conditions and ensure proper data sharing between the threads. The program should run for a specified number of iterations (e.g., 20), after which both threads should terminate gracefully.  Include error handling and clear comments in your code.


\section*{What to hand in (using Brightspace)}

1. Source file(s) with appropriate comments (e.g., \texttt{lab4\_problem2.c}, \texttt{lab4\_problem3.c}).
2. A single PDF file (\texttt{lab4.pdf}) containing:
    \begin{itemize}
    \item Screenshot(s) of your terminal window showing compilation, execution, and output for Problem 2 and Problem 3.
    \item The process tree diagram for Problem 2 (a).
    \end{itemize}
3. Makefile (if applicable).


\section*{RULES}

\begin{itemize}
\item Use kernel version 4.x.x or above.
\item Do not use kernel version 3.x.x.
\item Collaboration on general concepts is allowed; code copying is strictly prohibited.
\item Ask your TA for help if needed.
\item Submit your assignment before the deadline.
\end{itemize}


\end{document}