\documentclass{article}
\usepackage[utf8]{inputenc}
\usepackage{amsmath}
\usepackage{array}
\usepackage{graphicx}

\title{CS-UY 2214 — Recitation 1}
\date{}

\begin{document}

\maketitle

\section*{Introduction}
Complete the following 7 problems. Put your answers in a plain text file named \texttt{recitation1.txt}. Number your solution to each question. When you finish, submit your file on Gradescope. Then, in order to receive credit, you must ask your TA to check your work. Your work should be completed and checked during the recitation session.

Please note that your solutions must be in a plain text file. Other formats, such as PDF, RTF, and Microsoft Word, will not be accepted. Here are some recommended editors that produce plain text files:
\begin{itemize}
    \item Notepad (comes with Windows)
    \item TextEdit (comes with Mac OS); note that if you are using TextEdit, you need to select “Make Plain Text” from the Format menu before saving the file
    \item gedit (available on most Linux distributions)
    \item nano (available on most Linux distributions)
    \item Sublime Text
    \item VSCode
    \item Atom
    \item Vim
    \item Emacs
\end{itemize}

For questions that require a solution expressed as an image, submit the image as a separate file. The image file should be named \texttt{recitationnqm}, where $n$ is the recitation number and $m$ is the question number; use an appropriate suffix (either jpg or png).

\section*{Problems}
\noindent\textbf{1.}  Consider a combinational circuit with three inputs (A, B, C) and one output (Y). The output Y is 1 if and only if the majority of the inputs are 1.  (Insert Circuit Diagram Here as \texttt{recitation1q1.jpg} or \texttt{recitation1q1.png})

Express this circuit as a Boolean expression using only AND, OR, and NOT gates. Provide a truth table for this circuit.

\noindent\textbf{2.} Using only AND, OR, and NOT gates, construct a circuit diagram that will calculate the majority function with three inputs (A, B, C) and one output Y, where Y is 1 if at least two of the inputs are 1. Submit your answer as an image, in accordance with the instructions at the beginning of this document.  (Insert Circuit Diagram Here as \texttt{recitation1q2.jpg} or \texttt{recitation1q2.png})

\noindent\textbf{3.} Convert the following decimal numbers into binary.
(a) 47
(b) 128
(c) 255

\noindent\textbf{4.} Convert the following binary numbers into decimal.
(a) 1110
(b) 100000
(c) 11111111

\noindent\textbf{5.} Convert the following decimal numbers into hexadecimal.
(a) 255
(b) 1024
(c) 4095

\noindent\textbf{6.} Convert the following hexadecimal numbers into binary.
(a) cafe
(b) dead
(c) 1000

\noindent\textbf{7.}  Write a C++ program that takes two integer inputs from the user, representing the numerator and denominator of a fraction.  The program should then calculate and display the result of the division, handling potential division by zero errors gracefully.  Include error handling for non-integer input. The program should use a `try-catch` block to handle exceptions.


\end{document}