\documentclass{article}
\usepackage[utf8]{inputenc}
\usepackage{amsmath}
\usepackage{array}
\usepackage{graphicx}

\title{CS-UY 2214 — Recitation 1}
\date{}

\begin{document}

\maketitle

\section*{Introduction}
Complete the following 7 problems. Put your answers in a plain text file named \texttt{recitation1.txt}. Number your solution to each question. When you finish, submit your file on Gradescope. Then, in order to receive credit, you must ask your TA to check your work. Your work should be completed and checked during the recitation session.

Please note that your solutions must be in a plain text file. Other formats, such as PDF, RTF, and Microsoft Word, will not be accepted. Here are some recommended editors that produce plain text files:
\begin{itemize}
    \item Notepad (comes with Windows)
    \item TextEdit (comes with Mac OS); note that if you are using TextEdit, you need to select “Make Plain Text” from the Format menu before saving the file
    \item gedit (available on most Linux distributions)
    \item nano (available on most Linux distributions)
    \item Sublime Text
    \item VSCode
    \item Atom
    \item Vim
    \item Emacs
\end{itemize}

For questions that require a solution expressed as an image, submit the image as a separate file. The image file should be named \texttt{recitationnqm}, where $n$ is the recitation number and $m$ is the question number; use an appropriate suffix (either jpg or png).

\section*{Problems}
\noindent\textbf{1.} Consider the following circuit.  [Insert Circuit Diagram Here as \texttt{recitation1q1.jpg} or \texttt{recitation1q1.png}]  Assume the diagram shows a circuit with two inputs, A and B, and one output, Y.  The circuit consists of two AND gates, one OR gate, and one NOT gate.  The first AND gate has inputs A and B, the second AND gate has inputs A and the output of the NOT gate applied to B. The outputs of the two AND gates are connected to the inputs of the OR gate.

Express the circuit as a Boolean expression, using only AND, OR, and NOT. Provide a truth table for this circuit.

\noindent\textbf{2.} Using only AND, OR, and NOT gates, construct a circuit diagram that will calculate a 2-bit adder. Your circuit will have four inputs (two bits for each input number) and two outputs (one for the sum and one for the carry). Submit your answer as an image, in accordance with the instructions at the beginning of this document. Your image may be a diagram created in a drawing program, or it may be a photograph of a hand-drawn diagram on paper. [Insert Circuit Diagram Here as \texttt{recitation1q2.jpg} or \texttt{recitation1q2.png}]

\noindent\textbf{3.} Convert the following decimal numbers into binary.
(a) 47
(b) 128

\noindent\textbf{4.} Convert the following binary numbers into decimal.
(a) 1110
(b) 100000

\noindent\textbf{5.} Convert the following decimal numbers into hexadecimal.
(a) 51
(b) 255

\noindent\textbf{6.} Convert the following hexadecimal numbers into binary.
(a) cafe
(b) 1a2b

\noindent\textbf{7.} Convert the following decimal numbers into 8-bit binary 2’s complement. Write all the digits.  If a number cannot be represented, explain why.
(a) 100
(b) -100


\end{document}